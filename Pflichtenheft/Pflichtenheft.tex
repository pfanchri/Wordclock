\documentclass[11pt,a4paper,ngerman]{article}
 \usepackage[utf8]{inputenc} %UTF8 input file
\usepackage[T1]{fontenc} 
\usepackage[ngerman]{babel} %Umlaute,Silbentrennung 
\usepackage{booktabs}
\date{\today}
\author{Christof Pfannenmüller}
\title{DIY: Pflichtenheft Wordclock}
\begin{document}
 \maketitle
\tableofcontents 
\section{Über dieses Dokument}
Um die Entwicklung meiner Wordclock etwas zu ordnen möchte in diesem Dokument das Erstellen und die Aufgaben dabei etwas sortiert aufführen.
\section{Lastenheft}
{\large Hier habe ich das bereits erstellte Lastenheft eingefügt.  }\\
Lastenheft Wordclock:\\
\\
Weckfunktion: vorsehen einer Weckfunktion mit Ton und Snoozefunktion\\
\\
Taster zum Ausschalten der Alarmfunktion, Snozze und zum Einstellen der Uhrzeit\\
\\
Helligkeitsanpassung: über Sensor oder über Software und Schalter an der Rückseite\\
\\
Sekundenanzeige\\
\\
Datum und Wochentag anzeigen\\
\\
Funkanbindung zum automatischen Uhrzeit einstellen\\
\\
Stromreserve: Uhrzeit bleibt bestehen bei Stromausfall  evtl. auch Anzeige und Weckfunktion ohne Strom\\
\\
Ausschalter: kleinen Ausschalter an der Unterseite zum gänzlichen Ausschalten (Uhrzeit läuft über Batterie weiter\\
\\
USB Versorgung\\
\\
24h Modus: Wecker klingelt nur alle 24h nicht alle 12 und die Weckfunktion kann somit angelassene werden\\
\\
verschiedene Frontcovers möglich: in verschiedenen Sprachen, dafür verschiedene Möglichkeiten der LED Ansteuerung\\
\\
Stabiles Gehäuse, schön klein\\
\\
Standfuß (standsicher)\\

\section{Pflichtenheft}



\begin{tabular}[t]{lllll}
	\toprule
	Aufgabe & Beschreibung & Wie umzusetzen  & bis & Erledigt \\
	\midrule
	Weckfunktion & & Lautsprecher/Tongeber, Taster für Snoozefunktion & & \\
	Weckzeitpunkt einstellen&& mehrere Taster, mind. 2 (Mode, Up) besser 3 (Mode, Up, Down); Mode kurz drücken zeigt aktuellen Weckzeitpunkt an, lange drücken um in Modus Weckzeitpunkt kommen; Unterscheidungsmöglichkeit AM/PM notwendig, 
	Uhrzeit einstellen & & Taster Up (Down) von Weckzeitpunkt verwenden; eigenen Taster um in Modus Uhrzeit einstellen zu kommen, oder mehrere Taster gleichzeitig drücken; wen gewünscht Möglichkeit für Datum einstellen (daraus Wochentag berechnen) \\

	\bottomrule
\end{tabular}






\end{document}