\documentclass[11pt,a4paper,ngerman]{article}
 \usepackage[utf8]{inputenc} %UTF8 input file
\usepackage[T1]{fontenc} 
\usepackage[ngerman]{babel} %Umlaute,Silbentrennung 
\usepackage{booktabs}
\usepackage{lscape}
\date{\today}
\author{Christof Pfannenmüller}
\title{DIY: Pflichtenheft Wordclock}
\begin{document}
 \maketitle
\tableofcontents 
\section{Über dieses Dokument}
Um die Entwicklung meiner Wordclock etwas zu ordnen möchte in diesem Dokument das Erstellen und die Aufgaben dabei etwas sortiert aufführen.
\section{Lastenheft}
{\large Hier habe ich das bereits erstellte Lastenheft eingefügt.  }\\
Lastenheft Wordclock:\\
\\
Weckfunktion: vorsehen einer Weckfunktion mit Ton und Snoozefunktion\\
\\
Taster zum Ausschalten der Alarmfunktion, Snozze und zum Einstellen der Uhrzeit\\
\\
Helligkeitsanpassung: über Sensor oder über Software und Schalter an der Rückseite\\
\\
Sekundenanzeige\\
\\
Datum und Wochentag anzeigen\\
\\
Funkanbindung zum automatischen Uhrzeit einstellen\\
\\
Stromreserve: Uhrzeit bleibt bestehen bei Stromausfall  evtl. auch Anzeige und Weckfunktion ohne Strom\\
\\
Ausschalter: kleinen Ausschalter an der Unterseite zum gänzlichen Ausschalten (Uhrzeit läuft über Batterie weiter\\
\\
USB Versorgung\\
\\
24h Modus: Wecker klingelt nur alle 24h nicht alle 12 und die Weckfunktion kann somit angelassene werden\\
\\
verschiedene Frontcovers möglich: in verschiedenen Sprachen, dafür verschiedene Möglichkeiten der LED Ansteuerung\\
\\
Stabiles Gehäuse, schön klein\\
\\
Standfuß (standsicher)\\
\begin{landscape}
\section{Pflichtenheft}


 
\begin{tabular}[]{p{2cm}p{2cm}p{10cm}p{2cm}p{2cm}}
	\toprule
	Aufgabe & Beschreibung & Wie umzusetzen  & bis & Status \\
	\midrule
	Weckfunktion & & Lautsprecher/Tongeber, Taster für Snoozefunktion & & \\
	Weckzeitpunkt einstellen & & mehrere Taster, mind. 2 (Mode, Up) besser 3 (Mode, Up, Down); Mode kurz drücken zeigt aktuellen Weckzeitpunkt an, lange drücken um in Modus Weckzeitpunkt kommen; Unterscheidungsmöglichkeit AM/PM notwendig \\
	Uhrzeit einstellen & & Taster Up (Down) von Weckzeitpunkt verwenden; eigenen Taster um in Modus Uhrzeit einstellen zu kommen, oder mehrere Taster gleichzeitig drücken; wen gewünscht Möglichkeit für Datum einstellen (daraus Wochentag berechnen) \\
	Anpassung der Helligkeit & Anzeige heller bei heller Umgebung & Sensor für Umgebungslicht mit passender Elektronik; Loch in Gehäuse; Regelung der LEDs durch PWM (kompliziert) evtl. Controller oder durch IC (dafür ADU notwendig) \\
	Sekunden-Anzeige & zeigt Sekunden auf der Matrix an & Knopf durch den in Sekundenmodus gewechselt wird; Anzeige der Sekunden auf der Matrix, evtl. nicht bei jedem Frontcover möglich (muss deaktiviert werden können); wo Sekunden gespeichert? \\
	Anzeige Datum und Wochentag & & Wie anzeigen? Wo gespeichert? evtl. nicht bei jedem Frontcover möglich (muss deaktiviert werden können) \\
	Funkuhr & Einstellen der Uhrzeit nach Atomuhr & DCF77 Verbindung um Uhrzeit automatisch zu finden und einzustellen.\\
	Stromreserve & & Uhrzeit bleibt erhalten und Wecker klingelt trotz fehlendem Strom; Anzeige trotzdem aktiv würde Speicherdauer verkürzen ( Anzeige nur im Moment des Weckens oder gar nicht)\\
\end{tabular}
	%Seitenumbruch in Tabelle (besser lösen?)
	\newpage
	\begin{tabular}[]{p{2cm}p{2cm}p{10cm}p{2cm}p{2cm}}
	\toprule
	Aufgabe & Beschreibung & Wie umzusetzen  & bis & Status \\
	\midrule
	Uhr komplett ausschalten && Uhr über kleinen Schalter komplett ausschalten; Taster an Unterseite oder hinter Frontcover verstecken (bleibt Uhrzeit und Einstellungen erhalten?)\\
	Stromversorgung && Uhr wird über USB-Netzteil versorgt; USB Buchse verbauen \\
	24h Modus && Wecker klingelt nur um 6:00am, Uhr zeigt aber gleiches an für am/pm\\
	verschiedene Frontcovers & & Frontcover auswechselbar; verschiedene Farben möglich (evtl. bei Farben aktuelle Uhrzeit unterschiedlich gut erkennbar); verschiedene Sprachen (Umschalten der Matrix auf andere Sprache muss möglich sein, dafür Taster etc.)\\
	Stabiles Gehäuse & & Gehäuse mit sicherem Stand; stabil um nicht kaputt zu gehen (herunterfallen); vorerst aus Holz später Alu etc (aus einem Teil)\\
	Programmieren der Uhr && Schnittstelle um Uhr auch später noch programmieren zu können; an Unterseite oder hinter Frontplatte verstecken\\
\end{tabular}
\end{landscape}

\section{Umsetzung}
	Eine Fronplatte







\end{document}