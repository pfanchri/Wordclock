\documentclass[12pt,journal,compsoc]{IEEEtran}
\usepackage[utf8]{inputenc}
\usepackage[ngerman]{babel}
%\usepackage{tikz, tikzscale, adjustbox,enumerate, tabularx, multirow, booktabs}
\usepackage{graphicx, tabularx, booktabs, enumerate, multirow}
\usepackage{cite, textcomp}

\usepackage{listings}
\lstset{
  language=C,
  basicstyle=\ttfamily\small,
  keywordstyle=\color{blue}\ttfamily,
  stringstyle=\color{red}\ttfamily,
  commentstyle=\color{green}\ttfamily\normalsize,
  breakatwhitespace=false,
  breaklines=true
}


\makeatletter
{\obeylines\gdef\bt@eol{^^M}}
\newenvironment{breakabletexttt}
  {\ttfamily\hfuzz=0.4em
   \list{}{\leftmargin=2em
           \itemindent=-\leftmargin
           \listparindent=-\leftmargin
           \parsep=0pt}
   \item\relax\obeylines\obeyspaces\expandafter\breakable@texttt\@gobble}
  {\endlist}
\def\breakable@texttt{\futurelet\@let@token\breakable@texttti}
\def\breakable@texttti#1{%
  \ifx\@let@token\end
  \expandafter\end
  \else
    \expandafter\ifx\bt@eol\@let@token
      \par
    \else
      \string#1\hskip1sp
    \fi
    \expandafter\breakable@texttt
  \fi}
\makeatother



\usepackage{tikz, tikzscale} % tiks für Zeichnungen

\usepackage[europeanresistors,europeaninductors,arrowmos]{circuitikz}
\usepackage{amsmath}

%\usepackage[siunitx]{circuitikz}
%\usepackage[ngerman]{varioref}% Definiert variable Referenzierungen
%\usetikzlibrary{dsp,chains}

\providecommand{\PSforPDF}[1]{#1}
\setcounter{secnumdepth}{1}

% NOTE: PDF hyperlink and bookmark features are not required in IEEE
%       papers and their use requires extra complexity and work.
% *** IF USING HYPERREF BE SURE AND CHANGE THE EXAMPLE PDF ***
% *** TITLE/SUBJECT/AUTHOR/KEYWORDS INFO BELOW!!           ***
\newcommand\MYhyperrefoptions{bookmarks=true,bookmarksnumbered=true,
pdfpagemode={UseOutlines},plainpages=false,pdfpagelabels=true,
colorlinks=true,linkcolor={black},citecolor={black},pagecolor={black},
urlcolor={black},

pdftitle={WOLF:PI Wake on Lan for RaspberryPi},
pdfsubject={WOLF:PI},
pdfauthor={Niklas Duda, Christian Hesse}}




% correct bad hyphenation here
%\hyphenation{op-tical net-works semi-conduc-tor}


\begin{document} %----------------------------------------------------------------------------Beginn des Dokuments
%
% paper title
% can use linebreaks \\ within to get better formatting as desired
\title{DIY Wordclock}

\author{Christof Pfannenmüller}% <-this % stops a space



  % for Computer Society papers, we must declare the abstract and index terms
  % PRIOR to the title within the \IEEEcompsoctitleabstractindextext IEEEtran
  % command as these need to go into the title area created by \maketitle.
%  \IEEEcompsoctitleabstractindextext{%
%  \begin{abstract}
  %\boldmath
  %The abstract goes here.
%  \end{abstract}
  % IEEEtran.cls defaults to using nonbold math in the Abstract.
  % This preserves the distinction between vectors and scalars. However,
  % if the journal you are submitting to favors bold math in the abstract,
  % then you can use LaTeX's standard command \boldmath at the very start
  % of the abstract to achieve this. Many IEEE journals frown on math
  % in the abstract anyway. In particular, the Computer Society does
  % not want either math or citations to appear in the abstract.

  % Note that keywords are not normally used for peerreview papers.
%  \begin{IEEEkeywords}
%  Computer Society, IEEEtran, journal, \LaTeX, paper, template.
%  \end{IEEEkeywords}}


% make the title area
\maketitle


% To allow for easy dual compilation without having to reenter the
% abstract/keywords data, the \IEEEcompsoctitleabstractindextext text will
% not be used in maketitle, but will appear (i.e., to be "transported")
% here as \IEEEdisplaynotcompsoctitleabstractindextext when compsoc mode
% is not selected <OR> if conference mode is selected - because compsoc
% conference papers position the abstract like regular (non-compsoc)
% papers do!
\IEEEdisplaynotcompsoctitleabstractindextext
% \IEEEdisplaynotcompsoctitleabstractindextext has no effect when using
% compsoc under a non-conference mode.


% For peer review papers, you can put extra information on the cover
% page as needed:
% \ifCLASSOPTIONpeerreview
% \begin{center} \bfseries EDICS Category: 3-BBND \end{center}
% \fi
%
% For peerreview papers, this IEEEtran command inserts a page break and
% creates the second title. It will be ignored for other modes.
%\IEEEpeerreviewmaketitle

%----------------------------------------------------------------------------------------------Beginn des  eigentlichen Textes

\section{Einführung}

\IEEEPARstart{W}{ordclocks} haben in der Bastler oder DIY-Szene eine große Verbreitung und bergen eine gewisse Faszination. 



\section{Aufbau der Wordclock}
Grundsätzlich sind die meisten Wordclocks sehr ähnlich aufgebaut. Fast immer wird eine Frontplatte durch 


%----------------------------------------------------------------------------------------------Ende des eigentlichen Textes

\nocite{*}
%\begin{thebibliography}{1}
%\bibliography{literatur.bib}
%\bibliographystyle{plain}
%\bibitem{IEEEhowto:kopka}
%H.~Kopka and P.~W. Daly, \emph{A Guide to {\LaTeX}}, 3rd~ed.\hskip 1em plus
 % 0.5em minus 0.4em\relax Harlow, England: Addison-Wesley, 1999.

%\end{thebibliography}




% that's all folks
\end{document}


