\documentclass{beamer}
\usepackage[T1]{fontenc} 
\usepackage[ngerman]{babel} %Umlaute,Silbentrennung
% \usepackage{beamerthemesplit} // Activate for custom appearance
\usepackage{cleveref}
\usepackage{graphicx}
\title{DIY Projektvortrag}
\author{Christof Pfannenmüller}
\date{\today}
\begin{document}

\frame{\titlepage} % -----------Anforderungen: 10 15 min 7 8 folien

\section[Outline]{}
\frame{\tableofcontents}

\section{mein Projekt}
\subsection{ \huge{ Wordclock auf einer Platine}}
\frame
{
  \frametitle{mein  Projekt}
{\Large Was soll\huge{te} das Projekt enthalten?\\}\par -----------hier
\begin{itemize}
\item Wordclock
\item Möglichkeit mich aufzuwecken
\item klein genug für den Nachttisch -> max. 20cm
\item Stromreserve über die Nacht
\item viele viele verschiedene Sprachen
\item und noch mehr Varianten
\end{itemize}

}
 \frame{
\frametitle{Was ist eine Wordclock}
\begin{figure}
     \centering
     \includegraphics[width=.7\textwidth]{Wordclock1}
     \caption{eine Wordclock}
\end{figure}
}
\frame{
\begin{figure}
     \centering
     \includegraphics[width=.7\textwidth]{Wordclock2}
     \caption{eine andere Wordclock}
\end{figure}
}
\frame{
\frametitle{verschiedene Frontplatten}
An eine Wordclock kann man verschiedene Frontplatten anbringen\\
=> andere Sprachen und Farben 
\begin{figure}

     \includegraphics[width=0.3\textwidth]{Q2C_BIT_JP}
	 \includegraphics[width=0.3\textwidth]{Q2C_BIT_AR}
	  \includegraphics[width=0.3\textwidth]{Q2C_BIT_D3}
	   \includegraphics[width=0.3\textwidth]{Q2C_BIT_NL}
\end{figure}
}
\frame{
\frametitle{verschiedene Frontplatten}

\begin{figure}
\centering
     \includegraphics[width=\textwidth]{Slide_1_sprachen}
\end{figure}
}

\frame{
\frametitle{verschiedene Frontplatten}

\begin{figure}
\centering
     \includegraphics[width=\textwidth]{Slide_2_sprachen}
\end{figure}
}
\frame{
\frametitle{verschiedene Frontplatten}

\begin{figure}
\centering
     \includegraphics[width=\textwidth]{Slide_3_sprachen}
\end{figure}
}




\frame{
\frametitle{mit Frontplatten auch möglich}
\begin{figure}
     \centering
     \includegraphics[width=.7\textwidth]{Wordclock4}
     \caption{andere Funktion der Wordclock}
\end{figure}
}
\frame{
\frametitle{Funktion einer Wordclock}
\begin{figure}
     \centering
     \includegraphics[width=.7\textwidth]{Wordclock3}
     \caption{eine Wordclock}
\end{figure}
}

\frame{
\frametitle{Funktion einer Wordclock }
\begin{figure}
     \centering
     \includegraphics[width=.52\textwidth]{Matrix}
     \includegraphics[width=.52\textwidth]{einzellneLEDs}

     \caption{Matrix vs. Einzelansteuerung}
\end{figure}

}


\frame{
\frametitle{Stromreserve}
\begin{figure}[t]
     \includegraphics[width=.2\textwidth]{USB}
\end{figure}
\begin{itemize}
\item Stromversorgung über USB
\item Speicher um über die Nacht zu kommen\\
(wenn man mal den Stecker zieht)
\end{itemize}

}
\frame {\huge noch Fragen?}
\end{document}
